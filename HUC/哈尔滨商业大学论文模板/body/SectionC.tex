\section{公式测试}这节用来展示公式,交换图等。
 
\subsection{行内公式}
典范的同态$\lim_{\leftarrow F} W_r(S)\rightarrow \lim_{\leftarrow F} W_r(S/\pi S )$是同构。

\subsection{整行公式}
$$\mathbb{A}_{inf}=W(S^\flat)\cong \lim_{\leftarrow F} W_r(S)$$

\subsection{多行公式}
\begin{sloppypar}
多行公式的情况非常多,对齐与换行的要求也各不相同。所以选择合适的环境非常重要。这份文档里无法涵盖所有情况,所以提供一个教程用以参考:\url{http://blog.csdn.net/yanxiangtianji/article/details/54767265}
\end{sloppypar}



\subsubsection{align环境}
\begin{align*}
    \operatorname{E} (Z_{n+1} - Z_n | X_1,..., X_n)
    &= \operatorname{E} (S_{n+1}^2 - (n+1) \sigma^2 - S_n^2 + n \sigma^2 | X_1,..., X_n) \\
    &= \operatorname{E} (S_{n+1}^2 - S_n^2 - (n+1) \sigma^2 + n \sigma^2 | X_1,..., X_n) \\
    &= \operatorname{E} (X_{n+1}(X_{n+1} + 2\sum_{i=1}^n X_i) - \sigma^2 | X_1,..., X_n) \\
    &= \operatorname{E} (X_{n+1}X_{n+1})
       + 2\operatorname{E} (X_{n+1}) \sum_{i=1}^n X_i - \sigma^2 \\
    &= \sigma^2  - \sigma^2 =0.
\end{align*}

\subsubsection{split环境(内嵌)}
\begin{equation*}
    \begin{split}
    (a + b)^4
      &= (a + b)^2 (a + b)^2      \\
      &= (a^2 + 2ab + b^2)
         (a^2 + 2ab + b^2)        \\
      &= a^4 + 4a^3b + 6a^2b^2 + 4ab^3 + b^4
    \end{split}
\end{equation*}

\subsubsection{带大括号的多行公式}

\paragraph{cases}
$$
    f=
    \begin{cases}
      x + y = z,  \\
      1 + 2 = 3.  \\
    \end{cases}
$$

\paragraph{array}
$$ F^{HLLC}=\left\{
\begin{array}{rcl}
F_L       &      & {0      <      S_L}\\
F^*_L     &      & {S_L \leq 0 < S_M}\\
F^*_R     &      & {S_M \leq 0 < S_R}\\
F_R       &      & {S_R \leq 0}
\end{array} \right. $$
    
\paragraph{aligned}
\begin{equation}
    \left\{
     \begin{aligned}
     \overset{.}x(t) &=A_{ci}x(t)+B_{1ci}w(t)+B_{2ci}u(t)  \\
     z(t) &=C_{ci}x(t)+D_{ci}u(t) \\
     \end{aligned}
     \right.
\end{equation}
%\begin{equation}\label{eq:2}
%\sum_{i=0}^{\infty} a_i x^i
%\end{equation}
%首先通过vref命令来引用等式\vref{eq:2},eref也可以引用式\eqref{eq:2}

\begin{equation}
E = mc^{2}
\label{eq:1}
\end{equation}
在式(\ref{eq:1})的质能方程中$m$表示物体的质量。

\subsection{交换图}
\begin{sloppypar}
强烈推荐tikzcd-editor:\url{https://github.com/yishn/tikzcd-editor}
\end{sloppypar}

\begin{center}
\begin{tikzcd}
    T
    \arrow[drr, bend left, "x"]
    \arrow[ddr, bend right, "y"]
    \arrow[dr, dotted, "{(x,y)}" description] & & \\
    & X \times_Z Y \arrow[r, "p"] \arrow[d, "q"]
    & X \arrow[d, "f"] \\
    & Y \arrow[r, "g"]
    & Z
\end{tikzcd}
\end{center}

\begin{center}
    \begin{tikzcd}[row sep=scriptsize, column sep=scriptsize]
        & f^* E_V \arrow[dl] \arrow[rr] \arrow[dd] & & E_V \arrow[dl] \arrow[dd] \\
        f^* E \arrow[rr, crossing over] \arrow[dd] & & E \\
        & U \arrow[dl] \arrow[rr] & & V \arrow[dl] \\
        M \arrow[rr] & & N \arrow[from=uu, crossing over]\\
        \end{tikzcd}
\end{center}

\begin{center}
\begin{tikzpicture}[commutative diagrams/every diagram]
    \node (P0) at (90:2.3cm) {$X\otimes (Y\otimes (Z\otimes T))$};
    \node (P1) at (90+72:2cm) {$X\otimes ((Y\otimes Z)\otimes T))$} ;
    \node (P2) at (90+2*72:2cm) {\makebox[5ex][r]{$(X\otimes (Y\otimes Z))\otimes T$}};
    \node (P3) at (90+3*72:2cm) {\makebox[5ex][l]{$((X\otimes Y)\otimes Z)\otimes T$}};
    \node (P4) at (90+4*72:2cm) {$(X\otimes Y)\otimes (Z\otimes T)$};
    \path[commutative diagrams/.cd, every arrow, every label]
    (P0) edge node[swap] {$1\otimes\phi$} (P1)
    (P1) edge node[swap] {$\phi$} (P2)
    (P2) edge node {$\phi\otimes 1$} (P3)
    (P4) edge node {$\phi$} (P3)
    (P0) edge node {$\phi$} (P4);
\end{tikzpicture}
\end{center}
