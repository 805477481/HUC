\section{注释与引用}这节用来展示注释与引用。

\subsection{注释——脚注与尾注}
\subsubsection{脚注}
\par 这里是脚注测试\footnote{1111111111}这里是脚注测试这里是脚注测试这里是脚注测试\footnote{2222222222}这里是脚注测试这里是脚注测试这里是脚注测试这里是脚注测试这里是脚注测试这里是脚注测试这里是脚注测试这里是脚注测试这里是脚注测试这里是脚注测试这里是脚注测试这里是脚注测试这里是脚注测试这里是脚注测试这里是脚注测试\footnote{3333333333}这里是脚注测试这里是脚注测试这里是脚注测试这里是脚注测试这里是脚注测试这里是脚注测试这里是脚注测试这里是脚注测试这里是脚注测试这里是脚注测试这里是脚注测试这里是脚注测试

\textcolor{red}{\textbf{\uline{注意!正如这份演示中所出现的情况,若该页(也就是本文档中的前一页)剩余空间不大,不足以显示足够多的文档与脚注,那么该段文字就会被移至下一页而留下空白。目前我们尚未找到解决的方法,所以如果遇到了这个问题,请修改排版,以留下足够大的空间。}}}

\subsubsection{尾注}
\par 这里是尾注测试\endnote{伴随着互联网的发展以及新的网络应用的出现,互联网用户由单纯的“读”网页,向“读、写”网页,共同建设互联网发展,由此网上产生了大量带有用户主观感情的数据,从这些带...}这里是尾注测试这里是尾注测试这里是尾注测试这里是尾注测试\endnote{尾注测试2}这里是尾注测试这里是尾注测试这里是尾注测试这里是尾注测试这里是尾注测试这里是尾注测试这里是尾注测试这里是尾注测试这里是尾注测试这里是尾注测试这里是尾注测试这里是尾注测试这里是尾注测试这里是尾注测试这里是尾注测试\endnote{尾注测试3}这里是尾注测试这里是尾注测试这里是尾注测试这里是尾注测试这里是尾注测试这里是尾注测试这里是尾注测试这里是尾注测试这里是尾注测试

\par \textcolor{red}{\textbf{\uline{注意!endnotes宏包并不支持hyperref,也就是无法通过点击文中尾注标号以跳转到尾注。当然,这在打印出来的文档中并不会造成任何影响。}}}
\par \textcolor{blue}{\textbf{\uline{提示:尾注出现在全文最后。为了区分脚注与尾注的编号,我们在尾注编号前加上了“尾注”二字。}}}

\subsection{文献引用的演示}
\par 本模板使用biblatex进行文献管理,这是一套相对较新的系统。另外,使用了hushidong制作的符合gb7714-2015标准的biblatex样式。在此对他的工作表示感谢,要完成这样的样式非常不容易。本模板中gb7714-2015.bbx与gb7714-2015.cbx即为他的作品,在这里打包发布以便使用。
\par 默认的bib文件位于~/reference/thesis-ref.bib,内容是由Wang Tianshu制作,在此仅作演示之用。关于bib文件的编写与管理请自行查找相关教程。
\par 下方的演示已经给出了正文中引用文献的基本方法,这与传统的cite命令是类似的。如有更多需求,请至\url{https://github.com/hushidong/biblatex-gb7714-2015}查找相关资料。
\par 文献\parencite{Yang_Hy200215}中提到xxxxxxx。
\par 文献\parencite{Joa1999}中提到yyyyyyy。
\par 文献\parencite{Altman1997}中提到zzzzzzz。
\par \textcolor{blue}{\textbf{\uline{本模板推荐使用parencite而不是cite命令,因为这样能与脚注所产生编号进行区分。}}}

