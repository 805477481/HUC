\begin{thebibliography}{99}
	\bibitem{ul1} Haarmann H. Language in ethnicity: A view of basic ecological relations[M]. Walter de Gruyter, 1986.
	
	\bibitem{ul2} Landweer M L. Indicators of ethnolinguistic vitality[J]. Notes on sociolinguistics, 2000, 5(1): 5-22.
	
	\bibitem{ul3} Jarvis S. Conceptual transfer in the interlingual lexicon[M]. Indiana University Linguistics Club Publications, 1998.
	
	\bibitem{ul4} Talmy L. Toward a cognitive semantics[M]. MIT press, 2000.
	 
	\bibitem{ul5} Jarvis S, Pavlenko A. Crosslinguistic influence in language and cognition[M]. Routledge, 2008.
	
	\bibitem{ul6} The East Asian welfare model: Welfare orientalism and the state[M]. Psychology press, 1998.
	
	\bibitem{ul7} Lewis M P, Simons G F, Fennig C D. Ethnologue: Languages of the world[M]. Dallas, TX: SIL international, 2009.
	
	\bibitem{ul8} Kushner E. English as global language: problems, dangers, opportunities[J]. Diogenes, 2003, 50(2): 17-23.
	 
	\bibitem{ul9} Upchurch M. Sounding the Alarm about Extinction of World's Languages[J]. Seattle Times, 2000-08-27, 2000.
	
	\bibitem{ul10} Swain A D, Guttmann H E. Handbook of human-reliability analysis with emphasis on nuclear power plant applications. Final report[R]. Sandia National Labs., Albuquerque, NM (USA), 1983.
	 
	\bibitem{ul11} Moriarty M. Globalization and Minority-Language Policy and Planning[M]//Globalizing Language Policy and Planning. Palgrave Macmillan, London, 2015: 9-23.
	
\end{thebibliography}