\usepackage{fancybox,fancyvrb,shortvrb} %也许需要的宏包
\usepackage[heading]{ctex} %用来提供中文支持
\usepackage{amsmath} %
\usepackage{amssymb} %数学符号,定理等环境相关宏包
\usepackage{amsthm}  %
\usepackage{graphicx} %插入图片所需宏包
\usepackage{adjustbox}  %也许需要的宏包
\usepackage{xspace} %提供一些好用的空格命令
\usepackage{tikz-cd} %画交换图需要的宏包
\usepackage{url} %更好的超链接显示
\usepackage{array} %表格相关的宏包
\usepackage{booktabs} %表格相关的宏包
\usepackage{caption} %实现图片的多行说明
\usepackage{float} %图片与表格的更好排版
\usepackage{ifthen}%这个宏包提供逻辑判断命令
\newboolean{first}%定义一个布尔变量用于判断是否为首页
\setboolean{first}{true}%设定fist变量初值为true
\usepackage{lastpage}                                           
\usepackage{layout}
\usepackage{pdfpages}
\usepackage{cite}

\usepackage{titletoc}%使所有目录后都有引导点
\titlecontents{section}[0pt]{\addvspace{2pt}\filright}
{\contentspush{\thecontentslabel\ }}
{}{\titlerule*[8pt]{.}\contentspage}
%==========================================================
%图标公式按章节显示
\renewcommand {\thetable} {\thesection{}.\arabic{table}}
\renewcommand {\thefigure} {\thesection{}.\arabic{figure}}
\renewcommand{\theequation}{\arabic{section}-\arabic{equation}}
\DeclareCaptionFont{zhten}{\songti\zihao{5}\selectfont}%设置图表字体为宋体五号
\captionsetup{font = zhten}
\captionsetup{labelsep=quad}%冒号换为一个空格

\usepackage{ulem} %更好的下划线

\usepackage[ top=3.5cm, bottom=2.5cm, left=3.0cm, right=2.5cm]{geometry} %设置页边距

\usepackage{fontspec}                   %设置字体需要的宏包
\setmainfont{Times New Roman}           %设置西文字体为Times New Roman
\setCJKmainfont{SimSun}                 %设置中文字体为宋体
\renewcommand{\normalsize}{\zihao{-4}\setlength{\baselineskip}{20pt}}  %设置正文字号为小四
\linespread{1.5} %1.5倍行距
\showboxdepth=5
\showboxbreadth=5 
\setcounter{secnumdepth}{5}                                                                                     %
\ctexset { section = { name={,},format={\centering \heiti \zihao {-2}\setlength{\baselineskip}{20pt}} } }         %
\ctexset { subsection = { name={,},format={\heiti \zihao {-3}\setlength{\baselineskip}{20pt}} } } %设置各级系统的编号格式
\ctexset { subsubsection = { name={,},format={\heiti \zihao {4}\setlength{\baselineskip}{20pt}} } }          %
\ctexset { paragraph = { name={,},format={\heiti \zihao {-4}} } }               %
\ctexset { subparagraph = { name={,},format={\heiti \zihao {-4}} } }           %

\usepackage[bottom,perpage]{footmisc}               %脚注,显示在每页底部,编号按页重置
\renewcommand*{\footnotelayout}{\zihao{-5}\songti}  %设置脚注为小五号宋体
\renewcommand{\thefootnote}{[\arabic{footnote}]}    %设置脚注标记为  [编号]
                      %脚注的反向超链接

\usepackage{fancyhdr}               %

%\renewcommand{\headrulewidth}{0pt}  %
%\lhead{哈尔滨商业大学学士学位论文}                            %
%\chead{}                            %将页眉页脚设置为:仅在右下角显示页码
%\rhead{\TitleCHS}                            %
%\lfoot{}                            %
\cfoot{\thepage}                            %
%\rfoot{\thepage}                    %

\lhead{}
\chead{哈尔滨商业大学本科毕业设计(论文)}{\zihao{5}\songti}
\rhead{}
%%=====================================================
%%双线页眉的设置
\makeatletter %双线页眉
\def\headrule{{\if@fancyplain\let\headrulewidth\plainheadrulewidth\fi%
		\hrule\@height 2.276pt \@width\headwidth\vskip1pt%上面线为1pt粗
		\hrule\@height 0.4pt\@width\headwidth  %下面0.5pt粗
		\vskip-2\headrulewidth\vskip-1pt}      %两条线的距离1pt
	\vspace{6mm}}     %双线与下面正文之间的垂直间距
\makeatother

\usepackage{xcolor} %彩色的文字

\usepackage[hidelinks]{hyperref} %各种超链接必备

\usepackage{footnotebackref}  


\newtheorem{theorem}{\heiti 定理}[section]      %
\newtheorem*{theorem*}{\heiti 定理}             %
\newtheorem{lemma}[theorem]{\heiti 引理}        %
\newtheorem*{lemma*}{\heiti 引理}               %
\newtheorem{corollary}[theorem]{\heiti 推论}    %
\newtheorem*{corollary*}{\heiti 推论}           %
\newtheorem{definition}[theorem]{\heiti 定义}   %
\newtheorem*{definition*}{\heiti 定义}          %
\newtheorem{conjecture}[theorem]{\heiti 猜想}   %将各种常用环境设置为中文
\newtheorem*{conjecture*}{\heiti 猜想}          %
\newtheorem{problem}[theorem]{\heiti 问题}      %
\newtheorem*{problem*}{\heiti 问题}             %
\newenvironment{solution}                       %
  {\renewcommand\qedsymbol{$\blacksquare$}      %
  \begin{proof}[\heiti \bf 解]}                 %
  {\end{proof}}                                 %
\renewcommand*{\proofname}{\heiti \bf 证明}     %

\allowdisplaybreaks %允许公式跨页显示

\clearpage

\newcommand{\makeapdx}{
	\clearpage
	%%\apdx{附录}
%%23333333333333333333333333333333333333333
%
%\apdx{调查结果}
%23333333333333333333333333333333333333333


\begin{appendix}
	\section*{附录1}
	x
	\clearpage
	\section*{附录2}
	y
\end{appendix}
}

\newcommand{\makeacknowledgement}{	 %
\clearpage                           %生成感谢,请勿改动
\input{./ending/acknowledgement.tex} %
}                                    %

%For Algorithm
\usepackage{algorithm}
\usepackage{algorithmicx}
\usepackage{algpseudocode}
\floatname{algorithm}{算法}
\renewcommand{\algorithmicrequire}{\textbf{输入:}}
\renewcommand{\algorithmicensure}{\textbf{输出:}}
\usepackage{bm}


%可能会需要在用自然语言描述算法步骤时使用的宏包
\usepackage{enumitem}

%表格单元格内换行
\newcommand{\tabincell}[2]{\begin{tabular}{@{}#1@{}}#2\end{tabular}}

